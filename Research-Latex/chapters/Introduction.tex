


\subsection{Abstract}

Since the beginning of digital computing, there has been a massive jump from the traditionally used console (bash terminal) to interface with our programs to the now-standard graphical user face we see on our personal computers, smart devices, and even home appliances. Gesture systems have been a concept in sci-fi films like Iron-man II (2010) and television series like ’black mirror’ (2011). Voice, Gesture, and facial recognition are now standard delivered by Microsoft’s and Apple's assistant ’Siri’ and ’Cortana’ which allows users to interact in a new way. We almost forget how fast and convenient our devices and are ranging from our flagship smartphones to our Desktop Personal Computers. Due to the rapid growth of the wide use of computers for consumers businesses are developing and incorporating new ways to interact with their devices. Gestures for communication can help people suffering from speech impediments or short hearing can interact via these new technologies I will diagnose. Gestures can be found in day-to-day devices such as your smartphone, tablet, personal computer and even used in McDonald’s Interactive order takeaways. Throughout this paper, I will explore the current state of the User interface based gestures used in modern society, how these technologies formed and defer from traditional methods used such as the keyboard and mouse combination, their accessibility, and the challenges to overcome. Finally, I review these technologies and access their prevalence and use in modern society and diagnose if gesture systems will increase or decrease in popularity in years to come. 


